\documentclass[12pt]{article}
\usepackage{natbib}
\usepackage{hyperref}

\title{I expected a Model T, but instead I got a loom:\\ Awaiting the second big data revolution}
\author{Mark Huberty\thanks{Prepared for the 2013 BRIE-ETLA
    Conference, September 6-7, Claremont California. This paper has
    benefited from extended discussions with Chris Diehl, David
    Gutelius, Joseph Reisinger, Sean Taylor, Sean Gerrish, Drew
    Conway, Cathryn Carson, Georg Zachmann, and John
    Zysman. All errors commmitted, and opinions expressed, remain
    solely my own.}}

\begin{document}
\maketitle
\begin{abstract}
``Big data'' has been heralded as the agent of a third industrial
revolution--one with raw materials measured in bits, rather than tons
of steel or barrels of oil. Yet the industrial revolution transformed not just how
firms made things, but the fundamental approach to value creation in
industrial economies. To date, big data has not achieved this
distinction. Instead, today's successful big data business models
largely use data to scale old modes of value creation, rather than
invent new ones altogether. In this way, today's big data landscape
resembles the early phases of the first industrial revolution, rather
than the culmination of the second a century later. Realizing the
second big data revolution will require fundamentally different kinds
of data, different innovations, and different business models than
those seen to date. That fact has profound consequences for the kinds
of investments and innovations firms must seek, and the economic,
political, and social consequences that those innovations portend.
\end{abstract}


\section{Introduction}
\label{sec:introduction}

We believe that we live in an era of ``big data''. Firms today
accumulate, often nearly by accident, vast quantities of data about
their customers, suppliers, and the world at large. Technology firms
like Google or Facebook have led the pack in finding uses for such
data, but its imprint is visible throughout the economy. The expanding
sources and uses of data suggest to many the dawn of a new industrial
revolution. The cheerleaders of this data revolution proclaim that
these changes, over time, will rival the industrial revolution in
scope and consequences for economic and social prosperity.

Yet this ``big data'' revolution has so far fallen short of its
promise. Precious few firms transmutate data into novel
products. Instead, most rely on data to operate, at unprecedented
scale, business models with long pedigree in the media and retail
sectors. Big data, despite protests to the contrary, is thus an
incremental change--and its revolution one of degree, not kind. 

The reasons for these shortcomings point to the challenges we face in
realizing the promise of the big data revolution. Today's advances in
search, e-commerce, and social media relied on the creative
application of marginal improvements in processing power and
storage. In contrast, tomorrow's hopes for transforming real-world
outcomes in areas like health care, education, energy, and other
complex phenomena pose scientific and engineering challenges of an
entirely different scale. 


\section{Big data will be big because it's big (and other conventions)}
\label{sec:convention-reality}

The present enthusiasm for big data starts from the empirical
observation that firms today can gather every more data, about an ever
wider variety of subjects, at ever-decreasing cost. Big data's
advocates claim that this data will become the raw material that,
together with cheap compute power, fundamentally transforms both
business operations and technological innovation. Taken together,
these changes herald a new industrial revolution that, like its
predecessor, will fundamentally change how we govern, work, play, and
live \citep{mayer2013big}.

This conventional wisdom is perhaps best captured by McKinsey \&
Company's 2011 report on big data \citep{mckinsey2011}. They begin
with a straightforward definition of what should constitute big data:
anything too large to analyze with conventional tools for data
storage, querying, and analysis. Firms, they argue, will increasingly
generate such data as part of normal business operations; and 
firms who best exploit this data for both operational efficiency and
product design will triumph over their competition.

In contrast to these optimistic pronouncements, exactly what firms
should do with this data is left uncertain. But, in a postmodern twist
on a world built on bits and algorithms, that is taken as part of big
data's promise. Firms have found countless uses for what McKinsey
terms ``digital exhaust''--the e-commerce server logs, customer
service call recordings, social media interactions, and other
data--that a firm generates as a byproduct of doing business. Amazon
uses such data to better target product recommendations. Wal-Mart uses
it to optimize their logistical operations and further squeeze
suppliers on price. Google uses it to target advertising more
effectively. Facebook and LinkedIn use it to build brand loyalty
through tying users ever more deeply to their friends and professional
contacts.

The plethora of uses firms have found for such pedestrian data
suggests a universe of promise for satisfying more ambitions
goals. Data might thereby surpass steel or oil as a raw
material. Given cheap processing and data storage, and powerful new
algorithms for mining the trail of exhaust left by billions of
internet users, the availability of nearly free raw materials suggests
enormous potential for economic and technological change. 


\section{A revolution constrained: data, potential, and value creation}
\label{sec:but-what-can}

Unfortunately for big data's advocates, insight hasn't proven as
amenable to mass-manufacture. Google, Facebook, and Amazon are
enormously successful businesses. But their business models remain
very parochial. They rely on centuries-old business models, albeit
operating at unprecedented scale, to generate huge revenues. This is
very far from the maximalist hype of a big data revolution that was
supposed to change the 21st century in the way that steam, steel, or
rail changed the 19th. Big data has so far made it easier to sell
things, target ads, or stalk long-lost friends or lovers. But it
hasn't yet fundamentally reworked patterns of economic life, generated
entirely new occupations, or radically altered relationships with the
physical world. Big data thus appears oddly self-referential: we
generate massive amounts of data in the process of buying, viewing, or
stalking; with which we then sell, display, or track. This is a
parochial world view for a movement that envisions revolution.

Understanding how we might get from here to there requires a better
understanding of how and why data--big or small--might create value in
a world of better algorithms and cheap compute capacity. Close
examination shows that firms have largely used big data to improve on
existing business models, rather than adopt new ones; and that those
improvements have relied data to describe and predict activity in
worlds largely of their own making. Where firms have ventured beyond
these self-constructed virtual worlds, the data have proven far less
useful, and products built atop data far more prone to failure. 

\subsection{What can data do?}
\label{sec:what-can-data}

Data can, in increasing order of difficulty, help describe today's
world, predict tomorrow's, or infer causal relations between actions
and outcomes. The proliferation of data has come with the suggestion
that we can now describe, predict, or infer heretofore complex
relationships without difficulty. Yet the experience of big data has
not borne that suggestion out. As big data has ventured beyond
advertising and retail, its attempts at valid descriptions,
predictions, and inferences have run into problems. Moreover, those
problems do not originate with a lack of sophisticated algorithms or
insights. Instead, they point to inadequacy of most of our most common
data. 

The most obvious big data products and services today offer the
trifecta. Google's ad service aggregates enormous volumes of web traffic
data to infer the relationship between web search results and relevant
advertising; and uses that inference to price and sell
advertising. Amazon's retail model studies how purchases of one
product correlate with others, and recommends new products to
customers based on the similarity of their past purchasing
behaviors. LinkedIn studies relations among employees' career
trajectories to predict and recommend likely professional
contacts. Each of these services rely on huge amounts of web traffic
and social network data to capitalize on weak signals about human
behavior. 

Unfortunately, these successes have not proven to be harbingers of
broader progress in describing the real world, inferring causal
connections, or predicting the future. Google's Flu service
provides a very relevant example. In 2008, Google announced a new
collaboration with the Centers for Disease Control (CDC) to track and report
rates of influenza infection. Historically, the CDC had monitored US
flu infection patterns through a network of doctors that tracked and
reported ``influenza-like illness'' in their clinics and
hospitals. But data from network took up to two weeks to reach the
CDC--a long time in a world confronting SARS or avian flu. Developing
countries with weaker public health capabilities faced even greater
challenges. Google hypothesized that, when individuals or their family
members got the flu, they went looking on the internet--via Google, of
course--for medical
advice. In a highly cited paper, \cite{ginsberg2008detecting} showed
that they could predict region-specific influenza infection rates in the United
States using Google search frequency data. Here was the true promise
of big data--that we capitalize on data to better understand the world around us. 

The subsequent history of Google Flu illustrates the shortcomings of
the big data revolution. Google Flu has failed twice since its launch,
during the 2009 and 2012 flu seasons. The patterns and reasons for
failure speak to the limits of prediction. In 2009, Google Flu
under-predicted flu rates during the H1N1 pandemic. Post-mortem
analysis suggested that the different viral characteristics of H1N1
compared with garden-variety strains of influenza likely meant that
individuals didn't know they had a flu strain, and thus didn't go
looking for flu-related information
\citep{cook2011assessing}. Conversely, in 2012, Google Flu
over-predicted influenza infections. Google has yet to discuss why,
but speculation has centered on the intensive media coverage of an
early-onset flu season, which may have sparked interest in the flu by
healthy individuals \citep{butler2013google}.

The problems experienced Google Flu provides a particularly acute
warning of the risks inherent in trying to predict what will happen in
the real world based on the exhaust of the digital one. Google Flu
relied on a map--a mathematical relationship between online behavior
and real-world infection. Google built that map on historic patterns
of flu infection and search. It assumed that such patterns would
continue to hold in the future. But there was nothing fundamental
about those patterns, and Google had limited capacity to check whether
they continued to hold, or to update its map in real time. Either a change in
the physical world--a new virus--or the virtual one--media
coverage--were enough to render the map inaccurate. The CDC's old
reporting networks out-performed big data when it mattered most.

\subsection{Locating the value in data}
\label{sec:locating-value-data}


The Google Flu example suggests the limits to big data as a source of
mass-manufactured insight about the real world. But Google itself, and
its fellow big-data success stories, also illustrate the shortcomings
of big data as a source of fundamentally new forms of value creation. Most
headline big data business models have used their enhanced capacity to
describe, predict, or infer in order to implement--albeit at
impressive scale and complexity--centuries-old business models. Those
models create value not from the direct exchange between consumer and
producer, but via a web of transactions several orders removed from
the creation of the data itself. Categorizing today's big data
business models based on just how far they separate data generation
from value creation quickly illustrates how isolated the monetary
value of firms' data is from their primary customers. Having promised
a first-order world, big data has delivered a third-order
reality. 

Realizing the promise of the big data revolution will require
a different approach. The same problems that greeted flu prediction
have plagued other attempts to build big data applications that
forecast the real world. Engineering solutions to these problems that
draw on the potential of cheap computation and powerful algorithms
will require not different methods, but different raw materials. The
data those materials require must originate from a first-order
approach to studying and understanding the worlds we want to
improve. Such approaches will require very different models of firm
organization than those exploited by Google and its competitors in the
first big data revolution.


\subsubsection{Third-order value creation: the newspaper model}
\label{sec:third-order-value}

Most headline big data business models do make much money directly from
their customers. Instead, they rely on third parties--mostly
advertisers--to generate profits from data. The actual creation and
processing of data is only useful insofar as it's of use to those
third parties. In doing so, these models have merely implemented, at
impressive scale and complexity, the very old business model used by
the newspapers they have largely replaced.

If we reach back into the dim past when solvent newspapers were viable
businesses (rather than hobbies of the civic-minded rich), we will
remember that their business model had three major components:

\begin{enumerate}
\item Gathering, filtering, and analyzing news
\item Attract readers by providing that news at far below cost
\item Profit by selling access to those readers to advertisers
\end{enumerate}

As the newspaper model matured, the selling of access became more and
more refined: newspapers realized that people who read the business
pages differed from those who read the front page, or the style
section. Front-page ads were more visible to readers than those buried
on page A6. Newspapers soon started pricing access to their readers
accordingly. Bankers paid one price to advertise in the business
section, clothing designers another for the style pages. This
segmentation of the ad market evolved as the ad buyers and sellers
learned more about whose eyeballs were worth how much, when, and
where.

Newspapers were thus third-order models. The news services they
provided were valuable in their own right. But readers didn't pay for
them. Instead, news was a means of generating attention and data,
which was only valuable when sold to third parties in the form of ad
space. Data didn't directly contribute to improving the headline
product--news--except insofar as it generated revenue could be plowed
back into news gathering. The existence of a tabloid press of dubious
quality but healthy revenues proved the weakness of the link between
good journalism and profit.

From a value creation perspective, Google, Yahoo, and other ad-driven
big data businesses are nothing more than newspapers at scale. They
too provide useful services (then news, now email or search) to users
at rates far below cost. They too profit by selling access to those
users to third-party advertisers. They too accumulate and use data to
carve up the ad market. The scale of data they have available, of
course, dwarfs that of their newspaper ancestors. This data, combined
with cheap computation and powerful statistics, has enabled
operational efficiency, scale, and effectiveness far beyond what
newspapers could every have managed. But the business model
itself--the actual means by which these firms earn revenues--is
identical.

\subsubsection{Second-order value creation: the retail model}
\label{sec:second-order-value}


Big-box retail ranks as the other substantial success for big
data. Large retailers like Amazon, Wal-Mart, or Target have very
effectively used data to optimize their supply chain, identify trends
and logistical issues ahead of time, and maximize the likelihood of
both initial sales and return business from their customers. 

But at the end of the day, these businesses remain retailers. Big data
may enable them to operate more efficiently. But that efficiency is in
service of a model of value generation--retail--that has existed for a
very long time. As with Google and ads, big data has enabled these
retailers to attain scale and complexity heretofore unimaginable. In
doing so, at least some of their profitability has come from market
power over suppliers, who lack the access and data the retailers
command. But the fundamental means by which they create value is no
different than it was fifty years ago.

Retailers are thus second-order big data models. Unlike third-order
models, the data they gather has a lot of direct value to the
retailer. They don't need to rely on third party purchasers to give
the data value. But the actual moneymaking transaction--the retail
sale of goods and services--remains separated from the uses of data to
improve operational efficiency. 

%% Where do things like search fit in. Companies like Ravel--making
%% better use of data that already exists. Differ from the models
%% below--where people set out to gather data and turned it into
%% products. But people pay for the services, specifically for the
%% service capability. 

%% Role of data in value generation: most firm value-add comes from
%% tools, not data (which already exists, albeit in difft
%% form). Ravel's also not per se new -- just a new take on what
%% Lexis, et al already do. So can ignore?

\subsubsection{First-order value creation: the opportunity}
\label{sec:first-order-value}

Second- and third-order models find value in data
several steps removed from the actual transaction that generates the
data. But, as the Google Flu example illustrated, that data may have
far less value when separated from its virtual context. Thus while
these businesses enjoy effectively free raw materials, the potential
uses of those materials are in fact quite limited. Digital exhaust
from web browsing, shopping, or socializing has proven enormously
useful in the self-referential task of improving future web browsing, shopping, and
socializing. But that success has not translated success at tasks far
removed from the virtual world that generated this exhaust. Digital
exhaust may be plentiful and convenient to collect, but it offers
limited support for understanding or responding to real-world problems.

First-order models, in contrast, escape the Flu trap by building atop
purpose-specific data, conceived and collected with the specific
intent of solving specific problems. In doing so, they too capitalize
on the cheap storage, powerful algorithms, and inexpensive compute
power that made the first wave of big data firms possible. But they do
so in pursuit of rather different problems.

First order products remain in their infancy. But some nascent
examples suggest what might be possible. IBM's Watson famously used
its natural language and pattern recognition abilities to win
Jeopardy!. But now IBM has adapted Watson to medical diagnosis. By
learning from disease and health data gathered from millions of
patients, Watson can improve the quality, accuracy, and efficacy of
medical diagnosis and service to future patients.\footnote{See
\cite{wired2013a} for early results of experiments showing that Watson
can improve the accuracy of cancer diagnoses.} Watson closes the data
value loop: patient data is made valuable because it improves patient
services, not because it helps with insurance underwriting or product
manufacturing or logistics or some other third-party service. The
diversity of disease and the varied ways in which it can express
itself means that Watson's big data capabilities help it improve on
human judgment that could never incorporate such data alone.

Premise Corporation\footnote{See \url{http://premise.is/}.}, a small startup, provides another
example. Premise produces very granular data on macroeconomic
aggregates like inflation, in regions of the world where official
statistics are unavailable or unreliable. It builds products and
services for financial firms, development agencies, and other clients
directly on that data. There is a direct link between data
generation and value, rather than an attenuated link through
second-order operational efficiency or third-order demographic or
commercial profiling. 

Optimum Energy (OE)\footnote{See \url{http://optimumenergyco.com/}.} provides a final example. OE monitors and
aggregates data on building energy use--principally data
centers--across building types, environments, and locations. That data
enables it to build models for building energy use and efficiency
optimization. Those models, by learning building behaviors across many
different kinds of inputs and buildings, can perform better than
single-building models with limited scope. Most importantly, OE
creates value for clients by using this data to optimize energy
efficiency and reduce energy costs.

These first-order business models all rely on data specifically
obtained for their products. They deploy sensors and data gathering
networks with specific hypotheses in mind, and build products directly
atop those hypothesis. Watson diagnoses disease with disease data;
Premise estimates inflation through specifically-targeted price data;
OE instruments data centers and then tunes those centers with the data
from those instruments.

This reliance on purpose-specific data contrasts with third-order
models that rely on data gathered for purposes wholly unrelated to the
task at hand--what big data conventional wisdom calls ``digital
exhaust.'' To use the newspaper example, third-order models
assume--but can't specifically verify--that those who read the style
section are interested in purchasing new fashions. Hence they sold ads
about fashions, rather than stocks or lawnmowers or funeral
services. But it was still a guess. Google's success stemmed from
closing this information gap a bit--showing that people who viewed web
pages on fashion were likely to click on fashion ads. But that gap remains--But again, the
data that supports this is data generated by processes unrelated to
actual purchasing--activities like web surfing and search or email
exchange. 


\subsubsection{Concept blurring}
\label{sec:concept-blurring}

We should not overstate the separation among these models. Google
arguably contains all three: a third-order ad business model that
relies, in part, on products dependent on first-order goods like email
with effective spam detection or maps with StreetView, tuned with
second-order processes to optimize customer satisfaction with those
products. Many retailers--particularly vertically-integrated
manufacturer-retailers-- gather customer data specifically to improve
the quality or variety of the products they sell.

But the distinction remains valid insofar as we're waiting for a big
data revolution that does more than sell us ads or optimize
operational efficiency. Most of the most promising products--the
genetic, medical, economic, and social products that will directly
improve our lives, rather than just serve to sell us stuff--rely on
the hypothesis that big data will permit new forms of first-order
value generation. Yet to date it has not. We should ask why. 

\subsection{The unrealized promise of unreasonable data}
\label{sec:unre-prom-unre}

We should remember the root of the claim about big data. That claim
was perhaps best summarized by \cite{halevy2009unreasonable} in what
they termed ``the unreasonable effectiveness of data''. Most have
taken that to mean that data--and particularly more data--are
unreasonably effective \textit{everywhere} But that mis-states the authors'
claims. They did not claim that more data was always better. Rather,
they argued that, for specific kinds
of applications, history suggested that gathering more data paid better
dividends than inventing better algorithms. Where data are sparse, or
the phenomenon under measurement noisy, more data allow a more
complete picture of what we are interested in. Machine translation
provides a very pertinent example: human speech and writing varies
enormously within one language, let alone two. Faced with the choice
between better algorithms for understanding human language, and more
data to quantify the variance in language, more data appears to work
better.\footnote{Not everyone is convinced. Peter Norvig, head of
  research at Google, had a very public dispute with the linguist Noam
Chomsky over whether progress in machine translation contributed
anything at all to our understanding of human language. See
\url{http://norvig.com/chomsky.html} for Norvig's account of this
dispute and a link to Chomsky's position.} 

For other applications, the ``bigness'' of data may not matter at
all. If I want to know who will win an election, polling a thousand
people might be enough. Relying on Twitter, in contrast, will probably
fail \citep{gayo2011limits,gayo2012wanted,huberty2013}. We are not,
and likely will never be, in the ``N=all'' world that infatuated
\cite{mayer2013big}. The internet today does not offer it--we are not
all online, and we are not ourselves online. Hence the idea that we
might understand humanity by looking only online is doomed to miss
wide swaths of both individuals and societies. It offers thus little
insight, and consequently more danger, than big data's advocates confess.

These limits on the usefulness of third-order data point to real
limits to the progress that today's big data revolution will make. To
date, we've often thought that we will get data for free, from the
day-to-day business operations that throw off so much data. In some
cases we might (particularly for industrial processes where the
distinction between real-world and virtual-world behavior is kind of
meaningless). But in most cases, the utopian dreams of a big data
world will require actually setting out to measure and model the real
world as it actually exists. 

Why, then, are second- and third-order models so successful? In part
because the downside risk of getting measurement or prediction wrong
is very low. It matters little at the margins if Amazon recommends the
wrong thing, or Google serves the wrong ad. It matters enormously if
Watson diagnoses the wrong disease. Most second- and third-order
models today can tolerate relatively high error rates. But the things
we hope for in the future cannot. 


\section{A loom, not a model T}
\label{sec:loom-not-model}

Big data therefore falls short of proclamation that it represents the
biggest change in technological and economic possibility since the
industrial revolution. That revolution, in the span of a century or
so, fundamentally transformed almost every facet of human life. An
English peasant living in 1800 enjoyed relatively few advantages over
his Roman predecessor of eighteen centuries prior. Textiles were still
woven by hand; foodstuffs were perhaps reliable, but not easily stored
nor necessarily very diverse; steel was valuable but expensive to
make; animal power provided most non-human labor. Water was not safe
to drink, particularly in urban areas. Transportation was slow, and
impossible in some seasons.

This peasant's Edwardian great grandchildren knew a very different
world: of rapid transportation over air, sea, and land; of plentiful
and cheap steel; of diverse and easily stored foodstuffs; of
machine-made textiles; of municipal sewers that purged cholera from
the cities; and of a world where the modern medicine
was rapidly rendering the scourges of polio, smallpox, measles,
and malaria treatable, if one contracted them at all. 

Having ranked big data with the industrial revolution, we find
ourselves wondering why our present progress seems so paltry in
comparison. Google differs only in scale, but not mode of value
creation, from a newspaper. Target still earns money by retail, not
through some novel manipulation of nature or man. The fundamental
advances in computation required to make these processes possible were
largely made during and shortly after the Second World War. 

% Delong quote from Mokyr: 
% [the] first Industrial Revolution... had little or no scientific base…. [It] created a chemical industry with no chemistry, an iron industry without metallurgy, power machinery without thermodynamics… pragmatic bodies of applied knowledge in which things were know to work, but rarely was it understood why they worked…. Moreover, even when things were known to work, they tended to be inflexible and slow to improve… difficult to remove bugs, improve quality, and make products and processes more user-friendly without a more profound understanding of the natural processes involved… 

We are in the position of someone who, in 1840, having been promised a
Model T, looks around and sees only looms. Those looms were, to be
sure, better than hand weaving. They made cloth cheaper, clothes more
plentiful. But the innovations that turned the hand weaving of
1815 into the power looms of 1830 weren't that radical. They were
mostly water-powered--a far cry from the giant steam- or
electricity-driven factories of the late 19th century. The cloth they
made wasn't dramatically different than what had been woven by
hand--more plentiful and cheaper, to be sure, but not substantially
different.  

Much of the value of the first industrial revolution was like
this--differences of degree, rather than kind. The advances in organic
chemistry, rapid personal transportation, shipping, air travel, pain
medication, and other products had to wait for the second industrial
revolution. That revolution saw the emergence of fundamentally
different kinds of firms. Rather than improve on pre-existing
processes, these firms invested in huge industrial research and
development operations to discover and then commercialize new
scientific discoveries. Those firms were matched by significant
government investment in basic research and development, particularly
in the United States and Germany. The talented tinkerers, craftsmen,
and inventors that built the first power looms were replaced by the
trained engineers, scientists, and managers of the major industrial
concerns. They, in turn, drew on the much more abstract abstract
advances in our fundamental understanding of physical, chemical, and
information processes. These changes were expensive, complicated, and
slow--so slow that John Stuart Mill despaired, as late as 1871, of
human progress.


\section{Consequences}
\label{sec:consequences}

We have a loom, but we envision the possibility of a Model T. Today,
we can see glimmers of that possibility in IBM's Watson, Google's
self-driving car, Nest's adaptive thermostats, and other technologies
deeply embedded in, and reliant on, data generated from and around
real-world phenomena. None rely on ``digital exhaust''. They do not
create value by parsing customer data or optiimzing ad click-through
rates (though presumably it could). They are not the product of a
relatively few, straightforward (if ultimately quite useful)
insights. Instead, IBM, Google, and Nest dedicated substantial
resources to studying natural language processing, large-scale machine
learning, knowledge extraction, and other problems. The resulting
products represent industrial synthesis of a series of complex
innovations, linking machine intelligence, real-time sensing, and
industrial design. These products are thus much closer to what big
data's proponents have promised--but their methods are a world away
from the easy hype about mass-manufactured insights from digital exhaust.


\section{Dystopia}
\label{sec:dystopic-future}


Each also, in its way, illustrates the dark side of the big data
revolution. Big data and the Silicon Valley culture in which it emerged have
suffered whithering criticism for overly optimistic
techno-utopianism.\footnote{See here in particular Alex Payne's
  ``Letter to a Young Programmer'', and his related talks, focusing on
  the insularity of modern investing and its extreme
  risk-aversion. See
  \url{https://al3x.net/2013/05/23/letter-to-a-young-programmer.html}.}
We forget too easily that social unrest, instability, and conflict
were fellow-travelers with the industrial progress of the 19th
century. Today, the promise of big data brings with it at least three
dystopias, all visible even in today's early stages of technological
change. Securing the gains of industrialization required careful
choices about politics and policy. Big data will require no less. 

\subsection{The Benthamite dystopia}
\label{sec:benthamite-dystopia}

\begin{quote}
  Yearly reminder: unless you're over 60, you weren't promised flying
  cars. You were promised an oppressive cyberpunk dystopia. Here you
  go.\\

  \vspace{12pt}
  Kyle Marquis, via Twitter
\end{quote}

First, a dystopia of abuse. During the first industrial revolution,
Jeremy Bentham proposed that we solve the prison problem with
technology, rather than better social policy. His Panopticon
``[ground] rogues honest'' with minimal human effort, allowing the
jailers to watch their inmate charges while leaving the inmates
unawares. Attempts to implement this innovation proved impractical
with 19th century technology.

But the confluence of cloud computing and big data services have
simplified the problem. The centralization of diverse data about large
numbers of people onto the servers of a limited number of
firms--usually for reasons having nothing to do with surveillance per
se--has made the Panopticon trivial. Indeed, the very technologies
that permit that data to serve better ads, predict consumer wants, or
anticipate new trends are equally well suited to a mass surveillance
state. 

Sadly, it appears such a dystopia has crept up upon us, at least in
the United States. The whistleblower disclosures about the NSA PRISM
program portray a sophisticated effort to access, mine, and act on the
communications traffic of correspondence between foreigners and
American citizens. Whether by design or accident, that program spied
on American citizens, possibly in violation of laws that prohibit the
NSA from doing so. Congressional attempts to defund or otherwise
curtail the program have, as of late 2013, been sparse and
unsuccessful.

Hence, in the US and barring a change of law, ``big data'' business
models focused on people and behavior--the second and third order
models in particular--will likely find themselves unwilling
participants in system that many US citizens regard as a gross
violation of their right to privacy. Citizens and governments overseas may think
twice before doing business with these firms, or corresponding with
those who do.\footnote{This has already begun. The European Union has
  advised companies looking for cloud services to consider non-US
  providers. Early reports estimated that the US could lose \$35-45
  billion dollars in contracts by 2015 due to foreigners withdrawing
  for PRISM-related reasons.\citep{babcock2013}. Brazil has openly
  stated the intent to insulate its internal communications from
  traffic passing through US networks \citep{bbc2013brazil}.} 

\subsection{The Downton Abbey dystopia}
\label{sec:downt-abbey-dyst}

Second, a dystopia of under-use. The origins of much of what we now
know as big data lie in another era of artificial intelligence. And
while we remain far from replicating consciousness on a chip, we've
done quite well abstracting, standardizing, and automating what
formerly were human activities. This process began with physical tasks
like manufacturing\footnote{The old joke about the factory applies
  here. A perfect factory is said to require only a man and a dog to
  staff it. The man feeds the dog, and the dog makes sure the man
  doesn't touch anything.} But it has now invaded formerly safe
white-collar professions like accounting or legal services. Amazon has
purchased a company that builds robots to automate its distribution
centers--moving automation further up the retail value
chain. Self-driving cars may, by the middle of the 21st century,
replace taxis. The ability to learn and replicate a larger
and larger share of both regular and (ostensibly) irregular means that
a greater portion of human labor can now be automated. Like the first
industrial revolution, that generates huge returns for those doing the
automating. But it ravages employment for those being automated. 

We have, of course, been through this before. The ``creative
destruction'' of successive technological revolutions took us from a
world where most of the population worked in agriculture, to one where
most of it worked in factories, to one where most are now employed in
services of some form. In each case, these transitions caused real
hardship for those whose jobs were rendered superfluous. But they
ultimately made all of society better off. 

The combination of big data and algorithms threatens to break this
chain of improvement. The industrial revolution endowed machines with
the capacity to perform repetitive physical tasks. To do so, those
machines embodied the knowledge of their human designers and
operators. But the operators remained a critical part of the process,
forming the crucial link between abstract engineering and physical
production. Big data now appears poised to displace the operators, in
both physical tasks and repetitive white collar work. As our capacity
to abstract mental tasks into codified steps improves, a larger and
larger share of formerly middle-class tasks will become the purview of
machines. 

What would remain of the people that used to perform this work? The
danger may lie in a kind of Downton Abbey society: one where, as in
late Victorian and early Edwardian England, a few very
wealthy individuals who can command the personal attention of a large
servant class.\footnote{\cite{lindert1983reinterpreting} and \cite{lindert2000three} show that
  income inequality in industrial England worsened over the course of
  the 19th century, before beginning a correction in the early
  20th. Inequality peaked in the late 1860s and early 1870s. The
  subsequent decline of the English gentry and their Downton Abbey
  existence occurred in part due to the growing expense of servant
  labor.} The Downton Abbey version of this world isn't
necessarily so bad. Unlike the reality of Edwardian domestic service,
which could be quite difficult and cruel, Downton Abbey supplied a
vision of a relatively benevolent upstairs class, and a sensitive and
thoughtful, if less well-off, downstairs class. Lord Grantham may be
a landed aristocrat, but at least within his household he comes off as
responsive to the needs of his staff and interested in the development
of his local community--so long as it doesn't require voting
Labour. Such households surely existed, but were far from the norm.

These changes may extend far beyond white collar middle management. As
\cite{rao2012} has argued, the second industrial revolution
transformed entrepreneurship as well. Where the first phase had seen
the proliferation of small-scale entrepreneurship, the second replaced
small owner-managers with a professionalized managerial class in large
industrial concerns. He argues that entrepreneurship is going through
just such a transformation. Vast riches from startups are, and always
have been, rare. Instead, venture capital sees startup exit via
acquisition--into a large and stable firm like Google or Facebook--as
the preferred path for most investments. Those acquisitions turn
entrepreneur-managers into product managers for one small niche of a
large enterprise. In doing so, it mitigates some startup risk. But, as
Rao points out, it also puts these entrepreneurs in the position of
being, effectively, labor. Nowhere is this more apparent than in
commodity startup incubators like YCombinator, for which
entrepreneurship is a pure volume play.

%% Something here about the fact that we'd once thought that industry
%% would yield lesiure time--but our productiity keeps going up, and
%% leisure time hasn't changed much (in the US, at least)

\subsection{The Lehman dystopia}
\label{sec:lehman-dystopia}

Finally, a dystopia of misuse. Finance has a long lead on first-order
business models for big data. Financial firms have eagerly sought data
specifically for the purpose of building or refining financial
products. That dates to well before computers--the house of Rothschild
profited immensely from having received early warning of the outcome
at Waterloo, courtesy carrier pigeons. More recently, financial firms
have led their peer institutions in investing in information
technology and sensor systems to improve the quality, quantity, and
speed of data gathering. Entire classes of products--from
high-frequency trading to derivatives and crop insurance--rely on such
data and data analysis capabilities.

Yet in 2008, this system, for all its analytic prowess, utterly
failed. For a few days that September, normally sober people in
high office openly wondered whether the financial system would
collapse and take the rest of the economy with it. Several years
later, JP Morgan Chase found itself liable for a billion dollars in
reglatory fines, courtesy a faulty risk management model that exposed
the bank to several billion dollars in related losses.\citep{kopecki2013} Around the same time,
Knight Capital lost a half-billion dollars in less than an hour
consequence of accidentally using out-of-date computer models for
high-frequency trading \citep{strasburg2012}. More information, faster, does not seem to
brought stability to even the largest and most sophisticated institutions.

Most of the blame here lies in a combination of failed
macro-prudential regulation and very cheap money. But the misuse of
data plays an important part as well. Firms, having built
sophisticated models of risk atop reams of historical data, thought
themselves protected against the ravages of the market. Yet the very
data that enabled their products had simultaneously blinded them to
their extraordinary risk exposure. Whether the ensuing failures were the
product of willful misdeeds or na\"ive assumptions does not avoid the
fact that data lay at their core.

The Lehman dystopia is thus a dystopia of misuse: of data-driven
models that fail to deliver on their promise, consequence of
poorly-designed, badly-executed ideas. It's a dystopia wherein
organizations want the upside of big data without thinking too hard
about where it might go wrong. And it's a dystopia of yes-men, where
data becomes a means of confirming one's own biases--usually those
coincident with short-term profits--rather than a means of
understanding and operating in the world.

The danger here is already visible.  Many companies now claim to offer
``big data'' solutions that cut out the messy and expensive step of
hiring statisticians and engineers--and the inconvenience of listening
to them afterwards.\footnote{``Big data'' tools range from generic
technologies designed to handle data, such as Hadoop; to tools highly
tuned to specific problems, such as HortonWorks' server log processing
technology. These tools facilitate, but don't promise to replace,
human judgement in quantitative inference. Other companies, such as
Tableau, promise to make big data accessible through interfaces that
look like common spreadsheets. In an essay that bordered on parody,
\cite{mehta2013} claimed that these tools would commodify the
production of quantitative knowledge.} Instead, they promise to put
big data in the directly in the hands of day-to-day business
operations. But doing so is the equivalent of asking for the outputs
of an industrial research lab without the investment. These tools can
no more turn marketing people into data scientists than a chemistry
set would have turned a 19th century craftsman into an industrial dye
chemist. These business models--data science as a service, if you
will--will almost certainly fail substantively, even if they profit in
the short term.. We can only hope that they don't, as Lehman did,
cause widespread damage along the way.

%%%
These dystopias each have their precedent in the industrial revolution
to which big data is the supposed successor. Thyssen, Krupp, and IG
Farben were innovative industrial firms as well as the incubators of
the \textit{Wehrmacht} and Zyklon B. The early years of the industrial
revolution saw widespread wage declines and enormous displacement of
agricultural labor into the cities. We should remember that the
Luddites weren't idle layabouts, but skilled craftsmen angry at the
displacement of their hard-won knowledge and ability.  Consumer
products were a font of unreliability, danger, and malfunction before
product liability, testing, and standards emerged to prohibit the
worst of the snake oils and help verify the rest. 

% \subsection{The overtreatment dystopia}
% \label{sec:overtr-dyst}

% Finally, the ubiquity of data and analysis may confront us with a
% problem long known to medicine: over-monitoring leads to
% over-treatment, but does not obviously improve outcomes. Continuous
% fetal monitoring during childbirth increases the rate of medical
% intervention. Those interventions do not appear to improve outcomes
% for either mother or child--and in some cases are demonstrably worse
% than no intervention at all. Likewise, regular cancer screenings
% appear to lead to unnecessary medical interventions, particularly for
% very slow-growing varieties such as prostate cancer that pose little
% risk to the patient.  




% \section{Wherefore big data?}
% \label{sec:wherefore-big-data}

% Given the limitations, and potential abuses, of big data, wherefore
% the present enthusiasm? It may be due to a confluence of two
% factors. First, much of the big data discussion has, we have seen,
% gone awry for assuming that data from the online world contains
% answers to problems we face offline. Those enthusiastic about big data
% in all its forms thus appear to get the question backwards: they
% assume that we have data and the capacity to analyze it, and hence
% will find answers if we simply look hard enough. 

% But that is wrong, for the reasons this essay has taken pains to
% outline. We have few good reasons to assume that individuals' offline
% and online behaviors coincide perfectly. We have even less reason to
% believe that our algorithms are sufficiently discerning so as to
% detect when those behaviors diverge. Instead, the more operative
% question should be this: what questions do we wish to answer, and how
% can we employ cheap compute power, sensing, and storage to do so?

% That it's been so hard to re-frame the big data conversation may
% suggest something deeper. In many cases, big data appears to be a
% decidedly second-best solution to problems we once solved with
% government. Epidemiology, or (to take another favorite example),
% monitoring the safety of roadways and bridges, were for many years
% public functions. But the early 21st century is not a time of great
% faith in government. Hence the tasks we once left to bureaucrats to
% manage, we now try to solve with data that were never intended for the
% task. 

% Take infrastructure monitoring as one example. Today, the American
% Society of Civil Engineers estimates that US infrastructure will
% require \$3.6 trillion in investment by 2020 to bring the nation's
% dams, schools, roadways, bridges, and other infrastructure up to
% modern standards.\footnote{See
%   \url{http://www.infrastructurereportcard.org/}.} Bridges alone
% likely require twice as much funding as they presently receive. With
% so critical a problem, real-time monitoring of infrastructure would
% appear superfluous. But instead we talk of using real-time sensors to
% alert us to what we already know: that the state of the nation's
% infrastructure is sub-par, 

% \section{What role for government}
% \label{sec:what-role-government}



% This alone may explain why we so consistently assume that there must
% be answers to unposed questions hiding in big data. So many of those
% questions refer to problems, whether on crime or epidemiology or civil
% infrastructure, that we once referred to governments to handle. Take
% road and bridge safety monitoring. Today, if you wanted to know more
% about the sorry state of America's roads and bridges, you could ask
% the Federal Highway Authority. They would tell you that bridge funding
% is probably 50\% too low to keep infrastructure in an adequate state
% of repair. Failing that, you could ask the American Society of Civil
% Engineers, who would say that we're short \$3.6 trillion worth of
% infrastructure spending on roads, bridges, dams, schools, and other
% public facilities. 

% With numbers like that, we don't really need big data. 

\section{Awaiting the big data revolution}
\label{sec:awaiting-big-data}

We're stuck in the first industrial revolution. We have the power
looms and the water mills, but wonder, given all the hype, where we put the Model Ts and
telegraphs. The answer is a hard one. The big gains from big data will
require a transformation of organizational, technological, and
economic operations on par with that of the second industrial
revolution. Then, as now, firms had to invest heavily in industrial
research and development to build the foundations of entirely new
forms of value creation. Entirely new business models emerged, where
the first revolution marked only marginal changes of scope or scale. 






% Here: most things are basically doing (1) and (2), with some (3)
% thrown in to figure out how best to monetize it. Furthermore, the
% worlds that we do best are those of our own making--purchasing
% behavior, online social interaction, recommendation systems, and so
% for. Trying to use online data to do offline prediction has been
% fraught with problems--look at Google Flu. Even when online systems
% are used for crowd-sourced offline data, things start to look
% weird. See, for instance, Sean Taylor's paper on the distorting
% dynamics of human psychology on rating systems like Yelp. 

% Whether ``big data'' does any of these things in ways better or
% different from earlier ``small data'' depends heavily on the problem
% at hand. 

% Google Flu predicts rates of inGoogle Flu's predictions Something like Google Flu was impossible prior to the advent
% of large-scale search. But there's no particularly good reason that
% the Centers for Disease Control couldn't have gotten the same--or even
% better--results with an electronic reporting network.\footnote{In
%   fact, the CDC has since upgraded their reporting network, which has
%   significantly reduced the lag time between observing and reporting
%   influenza-like illnesses.} Google Flu was a second-best
% approach--interesting in the US, but most useful in
% less-developed countries with underdeveloped public health systems. 



% \subsection{Social versus industrial exhaust}
% \label{sec:industrial-exhaust}

% Most big data today is probably virtual--the server logs, browsing
% histories, and communication patterns that characterize how people
% behave online. In turn, most applications built on that data are
% themselves virtual--predicting what people want next from Amazon or who else
% they might want to connect with on LinkedIn. As we've seen, where this
% virtual data--the digital exhaust of online social processes--tries to
% go offline, it faces real challenges.

% Industrial data--output from manufacturing, used to detect, diagnose,
% and solve problems or improve efficiencies--faces fewer risks in this
% regard. Why? With social data, we have no ground truth--we don't
% actually know that people who look for flu information online actually
% have the flu. With industrial processes, though, we know that the
% ground truth exists and is stable--machines are built to do a specific
% thing, repeatedly. Humans are not. In other words, we cannot
% introspect into the minds of those whose behaviors we observe
% online. The assumptions we make to use data on those behaviors to
% predict or measure things are all vulnerable to being proven
% wrong--and we won't know they are wrong until our predictions go
% awry. In contrast, we know exactly what industrial systems are
% supposed to do, and need make no assumptions about whether that should
% continue to hold in the future. 



% \section{Why does big data create value?}
% \label{sec:why-does-big}

% Thus the reality of big data does not, at present, live up to the hype
% in its conventional wisdom. Yet well-known firms--Google, Amazon,
% Walmart, or Macy's--earn billions of dollars annually through business
% models that rely on huge datasets, massive computational power, and
% sophisticated algorithmic analysis. If these firms have succeeded,
% then why should we regard big data, so far, as hype?

% We start by posing a basic question: where exactly in these
% business models does ``big data'' actually generate profits? We will
% quickly see that there are various \textit{degrees of separation}
% between the actual generation of data and its role in value
% creation. Today, successful big data business models generate value
% several steps removed from the exchange at the heart of the data
% generating process. But the real, as yet unrealized promise of big
% data lies in first-order models that build valuable products directly
% atop data. 





% % 2nd / 3rd order models are self-referential--exhaust describes
% % behavior inside a world of their own making. Work so long as you
% % live inside that world, much harder outside it. 


\bibliography{/home/markhuberty/bibs/bigdata}
\bibliographystyle{apalike}
\end{document}