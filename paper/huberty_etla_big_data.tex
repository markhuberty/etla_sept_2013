\documentclass[12pt]{article}

\title{I expected a Model T, but instead I got a loom:\\ Awaiting the second big data revolution}
\author{Mark Huberty\thanks{Prepared for the 2013 BRIE-ETLA
    Conference, September 6-7, Claremont California. This paper has
    benefited from extended discussions with Chris Diehl, David
    Gutelius, Joseph Reisinger, Sean Taylor, Sean Gerrish, Drew
    Conway, Cathryn Carson, Georg Zachmann, and John
    Zysman. All errors commmitted, and opinions expressed, remain
    solely my own.}}

\begin{document}
\maketitle
\begin{abstract}
``Big data'' has been heralded as the agent of a third industrial
revolution. Yet the industrial revolution transformed not just how
firms made things, but the fundamental approach to value creation in
industrial economies. To date, big data has not achieved this
distinction. Instead, today's successful big data business models
largely use data to scale old modes of value creation, rather than
invent new ones altogether. In this way, today's big data landscape
resembles the early phases of the first industrial revolution, rather
than the culmination of the second a century later. Realizing the
second big data revolution will require fundamentally different kinds
of data, different innovations, and different business models than
those seen to date. That fact has profound consequences for the kinds
of investments and innovations firms must seek, and the economic,
political, and social consequences that those innovations portend.
\end{abstract}
\end{document}
